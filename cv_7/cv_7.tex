%%%%%%%%%%%%%%%%%%%%%%%%%%%%%%%%%%%%%%%%%
% "ModernCV" CV and Cover Letter
% LaTeX Template
% Version 1.3 (29/10/16)
%
% This template has been downloaded from:
% http://www.LaTeXTemplates.com
%
% Original author:
% Xavier Danaux (xdanaux@gmail.com) with modifications by:
% Vel (vel@latextemplates.com)
%
% License:
% CC BY-NC-SA 3.0 (http://creativecommons.org/licenses/by-nc-sa/3.0/)
%
% Important note:
% This template requires the moderncv.cls and .sty files to be in the same 
% directory as this .tex file. These files provide the resume style and themes 
% used for structuring the document.
%
%%%%%%%%%%%%%%%%%%%%%%%%%%%%%%%%%%%%%%%%%

%----------------------------------------------------------------------------------------
%	PACKAGES AND OTHER DOCUMENT CONFIGURATIONS
%----------------------------------------------------------------------------------------

\documentclass[12pt,a4paper,roman]{moderncv} % Font sizes: 10, 11, or 12; paper sizes: a4paper, letterpaper, a5paper, legalpaper, executivepaper or landscape; font families: sans or roman

\moderncvstyle{casual} % CV theme - options include: 'casual' (default), 'classic', 'oldstyle' and 'banking'
\moderncvcolor{blue} % CV color - options include: 'blue' (default), 'orange', 'green', 'red', 'purple', 'grey' and 'black'

\usepackage{lipsum} % Used for inserting dummy 'Lorem ipsum' text into the template
\usepackage[utf8]{inputenc} 
\usepackage{ragged2e}
\usepackage[scale=0.72]{geometry} % Reduce document margins
%\setlength{\hintscolumnwidth}{3cm} % Uncomment to change the width of the dates column
%\setlength{\makecvtitlenamewidth}{10cm} % For the 'classic' style, uncomment to adjust the width of the space allocated to your name

%----------------------------------------------------------------------------------------
%	NAME AND CONTACT INFORMATION SECTION
%----------------------------------------------------------------------------------------

\firstname{} % Your first name
\familyname{} % Your last name

% All information in this block is optional, comment out any lines you don't need
\title{Curriculum Vitae}
%\address{123 Broadway}{City, State 12345}
%\mobile{(000) 111 1111}
%\phone{(000) 111 1112}
%\fax{(000) 111 1113}
%\email{john@smith.com}
%\homepage{staff.org.edu/~jsmith}{staff.org.edu/$\sim$jsmith} % The first argument is the url for the clickable link, the second argument is the url displayed in the template - this allows special characters to be displayed such as the tilde in this example
%\extrainfo{additional information}
%\photo[70pt][0.4pt]{pictures/picture} % The first bracket is the picture height, the second is the thickness of the frame around the picture (0pt for no frame)
%\quote{"A witty and playful quotation" - John Smith}

%----------------------------------------------------------------------------------------

\begin{document}

%----------------------------------------------------------------------------------------
%	COVER LETTER
%----------------------------------------------------------------------------------------

% To remove the cover letter, comment out this entire block

\clearpage

%\recipient{HR Department}{Corporation\\123 Pleasant Lane\\12345 City, State} % Letter recipient
\date{\today} % Letter date
\opening{Dear Sir or Madam,} % Opening greeting
%\closing{Sincerely yours,} % Closing phrase
%\enclosure[Attached]{curriculum vit\ae{}} % List of enclosed documents

%\makelettertitle % Print letter title
\justify
\textit{14 de Octubre,2019}\newline\newline
\textit{Estimada Comisión Examinadora}\newline\newline
%% 
Escribo esta carta para expresar mi deseo de participar en el programa de intercambio 2020-I como estudiante de pregrado. Me defino como un estudiante motivado y con mucha pasión por la tecnología, pienso que en el mundo de la ingeniería no existe un límite para lo que uno puede realizar.

Invierto la mayor parte de mi tiempo tomando cursos en línea, leyendo libros, y aprendiendo nueva información acerca de ciencia y tecnología, en muchos casos incluso, profundizo los detalles que no son enseñados durante mis clases, mediante herramientas como Coursera, MIT OCW, EDX, simplemente por el gusto de aprender. Cuando las cosas no salen como lo planeado, no se tienen los recursos suficientes, o me enfrento a una situación limitante, solo tengo que recordar lo estupenda que es mi carrera para recuperar la motivación y continúo intentándolo.

Estudiar en el extranjero es uno de mis objetivos desde el inicio de mis estudios de pregrado, esto ya que al ser la ingeniería electrónica parte del núcleo de la tecnología y con gran impacto en millones de personas al rededor del mundo, es muy importante tener una visión amplia y poder ver otras realidades tecnológicas, sociales y económicas para así poder plantear mejores soluciones. La mejor forma de adquirir esta visión es experimentar en primera persona como se desarrolla este avance en distintos países, es allí donde considero que radica la importancia de este intercambio.

Además, adquirir esta nueva visión, estudiar en el extranjero me traerá beneficios tales como hacer incrementar mi red de contactos con estudiantes y profesores con experiencias distintas a la mía, esto representa posibilidades de colaboraciones e investigación esenciales para desarrollo académico y profesional de los miembros de ambas universidades.

Esta experiencia de intercambio me expondrá a una nueva cultura, estilo distinto de vida será una oportunidad única para expandir mis horizontes, mejorar mis habilidades interpersonales; me entusiasma la idea de conocer nuevas personas con formas diferentes de pensar, diferentes opiniones con las cuales podamos discutir sobre soluciones a problemas tecnológicos. Esto me permitirá crecer tanto personal como profesionalmente lo cual quedará reflejado en mi formación y futura carrera como ingeniero.

Muchas gracias de antemano por esta oportunidad y por considerar mi aplicación, confío que su programa se beneficiara mucho de mis habilidades y experiencia de igual manera este me ayudara a mejorar. Estoy convencido de que podre dar la talla en todos los retos que se me presenten en su universidad y hare todo lo posible para adaptarme a esta nueva etapa. Espero tener noticias suyas pronto.



\vfill
\textbf{Hanan Ronaldo Quispe Condori}
%\makeletterclosing % Print letter signature
\end{document}