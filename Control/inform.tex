%% Estructura principal para un reporte de Trabajos intersemanales CIRCAE %%
\documentclass[a4paper]{IEEEtran} %tamaño del papel y el tipo de transcripción que será IEEE
\usepackage[utf8]{inputenc} %el tipo de codificación que incluye símbolos como la tilde
\usepackage[spanish]{babel} % hacemos que nuestro documentación vaya en español
\usepackage{cite} % citas bibliográficas
\usepackage{graphicx} %gráficos, usaremos solo .jpg o .png con estándares que ya veremos
\usepackage{subfigure}
\usepackage{url}
\usepackage{amsmath}
\usepackage{booktabs} 
\providecommand{\keywords}[1]{\textbf{\textit{Términos Clave---}} #1}
\begin{document}
%\tableofcontents%tabla de contenidos
%\listoffigures%lista de figuras
\title{Control De Nivel De Brazo Levitado Por Hélice}
\author{Hanan Ronaldo Quispe Condori, CIRCAE Student Member}
%\markboth{INFORME CIRCAE 2019-08-05-G1-P3-001}{} % Codigo del informe que corresponde a: semestre | mes | dia | numero de grupo con la G antepuesta | numero de proyecto con la P antepuesta | número de informe
\maketitle
\begin{abstract}
El divisor de potencia de Wilkinson es un dispositivo pasivo con todos sus puertos emparejados, no tiene perdidas cuando el puerto de entrada se excita y los puertos de salida estan aislados, esta simulación implementará un divisor de potencia para el rango de frecuencias de 0 a 2GHz.
\end{abstract}

\section{Problema}
\label{sec:Problem}

\section{Fundamento Teórico}
\label{sec:fundamento}

\vspace{5mm}
\begin{equation}
[S]=
\begin{pmatrix}
S_{11}&S_{12}&S_{13}\\
S_{21}&S_{22}&S_{23}\\
S_{31}&S_{32}&S_{33}\\
\end{pmatrix}
\label{eq:scattering}
\end{equation}
Scattering matrix \ref{eq:scattering}.
\vspace{5mm}




\begin{equation}
\begin{split}
    S_{11}&=0 \\
    S_{22}&=S_{33}=0 \\
    S_{12}&=S_{21}=-\frac{j}{\sqrt{2}}\\
    S_{13}&=S_{31}=-\frac{j}{\sqrt{2}}\\
    S_{23}&=S_{32}=0
\end{split}
    \label{eq:parametros}
\end{equation}
La matriz de dispersión del modelo que se simulará es la siguiente
\begin{equation}
    \begin{pmatrix}
    0&-\frac{j}{\sqrt{2}}&-\frac{j}{\sqrt{2}}\\
    -\frac{j}{\sqrt{2}}&0&0\\
    -\frac{j}{\sqrt{2}}&0&0\\
    \end{pmatrix}
    \label{eq:matrix_numbers}
\end{equation}
\section{El modelo}

\begin{table}[h]
    \caption{Tabla de Parametros}
\begin{tabular}{@{}lll@{}}
\toprule
Parametro & Valor    & Descripción \\ \midrule
h         & 1.2 mm   & Grosor del Substrato \\
eps\_r    & 4.3      & Permitividad del Substrato \\
t         & 0.035 mm & Espesor de metalización \\ 
W50       & 2.35 mm  & 50 Ohms (Z0) Anchura de linea \\
W70       & 1.23 mm  & 70.71 Omhs (Z0$\sqrt{2}$) \\
l70       & 42.54 mm & Longitud de Lambda / 4 del ancho de línea Z0$\sqrt{2}$\\ \bottomrule
\end{tabular}
\label{tab:parametros_simulacion}
\end{table}



Una vez los parametros esten ingresados podremos usar sus valores usando sus nombres en cualquier momento de la simulación.

Usaremos los parametros para empezar a construir el divisor de potencia, utilizaremos las herramientas para modelado 3D y las transformaciones disponibles para lograr la geeometria deseada, se muestran imagenes de este proceso.


\section{Conclusiones}


Podemos ver en las curvas de los parametros de dispersión las caracteristicas de la matriz dada en el fundamento teórico, los parametros $$S_{31},S_{21}$$ que al ser iguales nos dicen que una division igual de potencia fue alcanzada.
Tambien se puede apreciar que os terminales de salida estan aislados esto se puede ver en la gráfica de los parametros $$S_{31},S_{21}$$ 

En la realización de esta simulación, se ha podido observar el fenomeno de división de potencia de este divisor, se entendieron conceptos los conceptos de matriz de dispersión, análisis par-impar, asi como tambien, el uso a un nivel básico de la herramienta de simulación CST Studio Suite.
%\vspace{10mm}
\bibliographystyle{ieeetr}
\bibliography{bibliografia}
\end{document}