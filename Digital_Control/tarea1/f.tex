\documentclass[12pt]{article}
\usepackage[spanish]{babel}
\usepackage{apacite}
\usepackage[utf8]{inputenc}
\usepackage{amsmath}
\usepackage{newtxtext,newtxmath}
\usepackage{listings}
\usepackage[usenames]{color}
\definecolor{gray97}{gray}{.97}
\definecolor{gray75}{gray}{.75}
\definecolor{gray45}{gray}{.45}
\definecolor{azul1}{RGB}{141,198,163}
\definecolor{azul2}{RGB}{24,107,122}
\definecolor{verde1}{RGB}{44,186,34}
\usepackage{textcomp}
\lstset{
		frame=Ltb,
		framerule=1pt,
		framextopmargin=5pt, %margen de arriba
		framexbottommargin=5pt, %margen de abajo
		framexleftmargin= -2pt, %separacion del margen izquierdo
		framesep=2pt,
		rulesep=0.2pt,
		backgroundcolor=\color{gray97},
		rulesepcolor=,
        tabsize=4,
        rulecolor=\color[RGB]{106, 182, 217}, %AZUL
        upquote=true,
        aboveskip={1.5\baselineskip}, %despues de la linea de texto
        columns=fixed,
        showstringspaces=false,
        extendedchars=true,
        breaklines=true,
        prebreak = \raisebox{0ex}[0ex][0ex]{\ensuremath{\hookleftarrow}},
        showtabs=false,
        showspaces=false,
        showstringspaces=false,
        basicstyle=\scriptsize\ttfamily\color[RGB]{39, 100, 46}, %Numeros de lineas, simbolos, puntos y coma y demas
        identifierstyle=\ttfamily\color[RGB]{56, 140, 189}, %variables
        commentstyle=\color[RGB]{62, 179, 101}, %comentarios
        stringstyle=\color[RGB]{247, 165, 42}, %impresiones
        keywordstyle=\bfseries\color[RGB]{237, 118, 150}, %funciones
        %
		numbers=left,
		numbersep=-7pt, %separacion del numero
		numberstyle=\tiny,
		numberfirstline = false,
		breaklines=true,
		}
\usepackage{graphicx}
\usepackage[colorinlistoftodos]{todonotes}
\usepackage{natbib} %citas bibliograficas estilo APA :p
\usepackage{eso-pic}
\usepackage{avant}
\usepackage[top=2cm,bottom=2cm,left=2.5cm,right=3cm,headsep=8pt,a4paper]{geometry}
\usepackage{fancyhdr}
%\pagestyle{fancy}
%\fancyhf{}
%\fancyhead[LE,RO]{}
%\fancyhead[RE,LO]{Robótica}
%\fancyfoot[CE,CO]{\leftmark}
%\fancyfoot[LE,RO]{\thepage}
\renewcommand{\headrulewidth}{2pt}
\renewcommand{\footrulewidth}{1pt}
\usepackage{tabu}
\usepackage{array}
\usepackage{multirow}
\usepackage{amssymb}
\usepackage{makeidx}
\graphicspath{ {images/} }
\usepackage{wrapfig}
\usepackage{enumerate}
\usepackage{amsmath,tikz}
\usetikzlibrary{matrix}
\usepackage{steinmetz}
\newcommand*{\horzbar}{\rule[0.05ex]{2.5ex}{0.5pt}}
\usepackage{calc}
\date{\today}


\begin{document}

\begin{titlepage}
\newcommand{\HRule}{\rule{\linewidth}{0.5mm}} 
\center
\textsc{\LARGE  Benemérita Universidad \\[0.2cm] Autónoma de Puebla}\\[1.5cm] 
%\includegraphics[width=4cm]{IMAGENES/escudo}\\[1cm]
\textsc{\Large Facultad de Ciencias de la Electrónica}\\[0.5cm] 
\textsc{\large Licenciatura en Electrónica}\\[0.5cm]
\HRule \\[0.4cm]
{ \huge \bfseries Cinemática}\\[0.4cm] 
\HRule \\[1.5cm]
\begin{minipage}{\textwidth}
\center 
\textsc{\LARGE Robótica}\\[1.7cm] 
\emph{Profesor:} \\
Dr. Fernando Reyes Cortés \\[1cm]
\begin{tabular}{ll}
\emph{Alumno:} & \emph{Número de Matrícula:}\\
Hanan Ronaldo Quispe Condori  & 555010653\\
\end{tabular}
\end{minipage}\\[2cm]
\today
\end{titlepage}

\newpage

\textbf{A)} Demostrar la Controlabilidad Completa del sistema.
\vspace{10mm}

Usaremos la función $ctrb$ para calcular directamente la matriz de controlabilidad del sistema dado seguidamente usaremos la función $rank$ para calcular el rango de esta matriz.
Sea el siguiente script de MATLAB
\lstinputlisting[language=Matlab]{tarea2.m}
El resultado de este script es el siguiente.
\lstinputlisting[language=Matlab]{diary1}

Tenemos el rango de la matriz de controlabilidad es $5$ por lo tanto queda demostrada la controlabilidad completa del sistema.
\vspace{10mm}

\textbf{B)} Demostrar la Observabilidad Completa del sistema.
\vspace{10mm}

Usaremos la función $obsv$ para calcular directamente la matriz de observabilidad del sistema dado seguidamente usaremos la función $rank$ para calcular el rango de esta matriz.
Sea el siguiente script de MATLAB
\lstinputlisting[language=Matlab]{a.m}
El resultado de este script es el siguiente.
\lstinputlisting[language=Matlab]{diary2}

Tenemos el rango de la matriz de observabilidad es $5$ por lo tanto queda demostrada la observabilidad completa del sistema.

\vspace{10mm}
\textbf{C)} Calcular la matriz de ganacias del Observador de Estado.
\vspace{10mm}

Usaremos la operaciones simbolicas para calcular esta matriz de ganancias, los valores propios deseados se encuentran en el vector P.
\lstinputlisting[language=Matlab]{observ.m}
En la linea 9 se usará la ecuación dada en la hoja de teoria.
\lstinputlisting[language=Matlab]{diary}
Apartir de este punto se tiene que calcular el determinante de la matriz de error e igualar este a a la matriz caracteristica con los polos deseados esta ecuación se encuentra en la linea 33, seguidamente se usara la función $solve$ para calcular las ganancias deseadas. para simplificar estas operaciones se uso la función $place$ dando la siguiente matriz de ganancias.
\lstinputlisting[language=Matlab]{diary3}
\end{document}