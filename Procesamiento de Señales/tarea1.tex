\documentclass[12pt]{article}
\usepackage[spanish]{babel}
\usepackage[utf8]{inputenc}
\usepackage{amsmath}
\usepackage{listings}
\usepackage[usenames]{color}
\definecolor{gray97}{gray}{.97}
\definecolor{gray75}{gray}{.75}
\definecolor{gray45}{gray}{.45}
\definecolor{azul1}{RGB}{141,198,163}
\definecolor{azul2}{RGB}{24,107,122}
\definecolor{verde1}{RGB}{44,186,34}
\usepackage{textcomp}
\lstset{
		frame=Ltb,
		framerule=1pt,
		framextopmargin=3pt,
		framexbottommargin=3pt,
		framexleftmargin=0.4cm,
		framesep=0pt,
		rulesep=.4pt,
		backgroundcolor=\color{gray97},
		rulesepcolor=,
        tabsize=4,
        rulecolor=\color{azul1},
        basicstyle=\scriptsize\ttfamily,
        upquote=true,
        aboveskip={1.5\baselineskip},
        columns=fixed,
        showstringspaces=false,
        extendedchars=true,
        breaklines=true,
        prebreak = \raisebox{0ex}[0ex][0ex]{\ensuremath{\hookleftarrow}},
        showtabs=false,
        showspaces=false,
        showstringspaces=false,
        identifierstyle=\ttfamily,
        keywordstyle=\color[rgb]{0,0,1},
        commentstyle=\color[rgb]{0.133,0.545,0.133},
        stringstyle=\color[rgb]{0.627,0.126,0.941},
        keywordstyle=\bfseries,
        %
		numbers=left,
		numbersep=15pt,
		numberstyle=\tiny,
		numberfirstline = false,
		breaklines=true,
		}
\usepackage{graphicx}
\usepackage[colorinlistoftodos]{todonotes}
\usepackage{natbib} %citas bibliograficas estilo APA :p
\usepackage{eso-pic}
\usepackage{avant}
\usepackage[top=2cm,bottom=2cm,left=2.5cm,right=3cm,headsep=8pt,a4paper]{geometry}
\usepackage{fancyhdr}
\pagestyle{fancy}
\fancyhf{}
%\fancyhead[LE,RO]{}
\fancyhead[RE,LO]{Procesamiento de Señales II}
\fancyfoot[CE,CO]{\leftmark}
\fancyfoot[LE,RO]{\thepage}
\renewcommand{\headrulewidth}{2pt}
\renewcommand{\footrulewidth}{1pt}
\usepackage{tabu}
\usepackage{array}
\usepackage{multirow}
\usepackage{amssymb}
\usepackage{makeidx}
\graphicspath{ {images/} }
\usepackage{wrapfig}
\usepackage{enumerate}
\usepackage{amsmath,tikz}
\usetikzlibrary{matrix}
\usepackage{steinmetz}
\newcommand*{\horzbar}{\rule[0.05ex]{2.5ex}{0.5pt}}
\usepackage{calc}
\date{\today}


\begin{document}

\begin{titlepage}
\newcommand{\HRule}{\rule{\linewidth}{0.5mm}} 
\center
\textsc{\LARGE  Benemérita Universidad \\[0.2cm] Autónoma de Puebla}\\[1.5cm] 
\includegraphics[width=4cm]{IMAGENES/escudo}\\[1cm]
\textsc{\Large Facultad de Ciencias de la Electrónica}\\[0.5cm] 
\textsc{\large Licenciatura en Electrónica}\\[0.5cm]
\HRule \\[0.4cm]
{ \huge \bfseries Tarea 1}\\[0.4cm] 
\HRule \\[1.5cm]
\begin{minipage}{\textwidth}
\center 

\emph{Profesor:} \\
Fernando \textsc{Lopez Marcos} \\[1cm]

\begin{tabular}{ll}
\emph{Alumnos:} & \emph{Numero de Matricula:}\\
Hanan Ronaldo \textsc{Quispe Condori } & 111111 \\
\end{tabular}
\end{minipage}\\[2cm]
\today
\end{titlepage}

%\newpage
%~\vfill
%\thispagestyle{empty}
%\begin{figure}[hbtp]


%\includegraphics[width=4cm]{IMAGENES/motordc}
%\end{figure}
%\noindent \textsc{Trabajo Encargado: Problemas en MatLab \\ Máquinas Eléctricas \\ Universidad Nacional de San Antonio abad del Cusco}\\
%noindent \textsc{Ingeniería Electrónica }\\
%\noindent \textit{Tercera revisión, \today}

%\tableofcontents indice bloqueado xD

\newpage
\section{Codigo}


{\lstinputlisting[language=Matlab]{Matlab/senoidales.m}
\begin{figure}[h]
    \centering
        \includegraphics[width=16cm]{IMAGENES/untitled}
        \caption{Valor de la secuencia usando la aproximación  de Matlab.}
        \label{fig:muestra_aprox_matlab}
\end{figure}
La segunda imagen
\begin{figure}[h]
    \centering
        \includegraphics[width=16cm]{IMAGENES/untitled1}
        \caption{Valor de la secuencia usando la aproximación  de 3.14.}
        \label{fig:muestra_aprox_propia}
\end{figure}

\end{document}