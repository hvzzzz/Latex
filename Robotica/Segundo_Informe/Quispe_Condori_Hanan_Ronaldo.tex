\documentclass[12pt]{article}
\usepackage[spanish]{babel}
\usepackage{apacite}
\usepackage[utf8]{inputenc}
\usepackage{amsmath}
\usepackage{newtxtext,newtxmath}
\usepackage{listings}
\usepackage[usenames]{color}
\definecolor{gray97}{gray}{.97}
\definecolor{gray75}{gray}{.75}
\definecolor{gray45}{gray}{.45}
\definecolor{azul1}{RGB}{141,198,163}
\definecolor{azul2}{RGB}{24,107,122}
\definecolor{verde1}{RGB}{44,186,34}
\usepackage{textcomp}
\lstset{
		frame=Ltb,
		framerule=1pt,
		framextopmargin=3pt,
		framexbottommargin=3pt,
		framexleftmargin=0.4cm,
		framesep=0pt,
		rulesep=.4pt,
		backgroundcolor=\color{gray97},
		rulesepcolor=,
        tabsize=4,
        rulecolor=\color{azul1},
        basicstyle=\scriptsize\rmfamily,
        upquote=true,
        aboveskip={1.5\baselineskip},
        columns=fixed,
        showstringspaces=false,
        extendedchars=true,
        breaklines=true,
        prebreak = \raisebox{0ex}[0ex][0ex]{\ensuremath{\hookleftarrow}},
        showtabs=false,
        showspaces=false,
        showstringspaces=false,
        identifierstyle=\rmfamily,
        keywordstyle=\color[rgb]{0,0,1},
        commentstyle=\color[rgb]{0.133,0.545,0.133},
        stringstyle=\color[rgb]{0.627,0.126,0.941},
        keywordstyle=\bfseries,
		numbers=left,
		numbersep=15pt,
		numberstyle=\tiny,
		numberfirstline = false,
		breaklines=true,
		}
\usepackage{graphicx}
\usepackage[colorinlistoftodos]{todonotes}
\usepackage{natbib} %citas bibliograficas estilo APA :p
\usepackage{eso-pic}
\usepackage{avant}
\usepackage[top=2cm,bottom=2cm,left=2.5cm,right=3cm,headsep=8pt,a4paper]{geometry}
\usepackage{fancyhdr}
\pagestyle{fancy}
\fancyhf{}
%\fancyhead[LE,RO]{}
\fancyhead[RE,LO]{Robótica}
\fancyfoot[CE,CO]{\leftmark}
\fancyfoot[LE,RO]{\thepage}
\renewcommand{\headrulewidth}{2pt}
\renewcommand{\footrulewidth}{1pt}
\usepackage{tabu}
\usepackage{array}
\usepackage{multirow}
\usepackage{amssymb}
\usepackage{makeidx}
\graphicspath{ {images/} }
\usepackage{wrapfig}
\usepackage{enumerate}
\usepackage{amsmath,tikz}
\usetikzlibrary{matrix}
\usepackage{steinmetz}
\newcommand*{\horzbar}{\rule[0.05ex]{2.5ex}{0.5pt}}
\usepackage{calc}
\date{\today}


\begin{document}

\begin{titlepage}
\newcommand{\HRule}{\rule{\linewidth}{0.5mm}} 
\center
\textsc{\LARGE  Benemérita Universidad \\[0.2cm] Autónoma de Puebla}\\[1.5cm] 
\includegraphics[width=4cm]{IMAGENES/escudo}\\[1cm]
\textsc{\Large Facultad de Ciencias de la Electrónica}\\[0.5cm] 
\textsc{\large Licenciatura en Electrónica}\\[0.5cm]
\HRule \\[0.4cm]
{ \huge \bfseries Cinemática}\\[0.4cm] 
\HRule \\[1.5cm]
\begin{minipage}{\textwidth}
\center 
\textsc{\LARGE Robótica}\\[1.7cm] 
\emph{Profesor:} \\
Dr. Fernando Reyes Cortés \\[1cm]
\begin{tabular}{ll}
\emph{Alumno:} & \emph{Número de Matrícula:}\\
Hanan Ronaldo Quispe Condori  & 555010653\\
\end{tabular}
\end{minipage}\\[2cm]
\today
\end{titlepage}

\newpage
\section{Introducción}
En el presente informe se desarrollaran soluciones mediante el software de MATLAB para sistemas dinámicos ademas de ello, se presentará un análisis cualitativo de los datos obtenidos 
en el laboratorio de robótica.\vspace{4mm}
\\ 
Estos sistemas dinámicos son modelos matemáticos de ecuaciones diferenciales que describen fenomenos físicos que se encuentren presentes en 
el robot[\cite{reyes2011robotica}].\vspace{4mm} 
\\
Haremos un análisis de las características cualitativas del punto de equilibrio del sistema dinámico(ecuación en lazo cerrado formada por la dinámica del péndulo-robot y el algoritmo de control)
,esto con la ayuda de los datos obtenidos del experimento realizado en el laboratorio.\vspace{4mm}
\\
Se verá que la implementación del código presentado en clase hizo más rapido el proceso de simulación y resolución de los problemas planteados, esto solo se pudo lograr luego de 
una compresión de los mismos.\vspace{4mm}
\\
Cada solución ofrecida a los problemas se dará de la manera 
más clara posible tal y como se afronto en la realización de los mismos.\vspace{4mm}
\\ 
Este informe consta de los siguientes apartados

\begin{itemize}
    \item Introducción
    \item Propositos
    \item Descripción del Problema
    \item Solución del Problema
    \item Resultados
    \item Conclusiones
\end{itemize}

Todos estos apartados se desarrollarán de la manera más concisa relacionada al problema propuesto.\vspace{4mm} 
\\
La solución mediante MATLAB es de gran utilidad, este al ser un lenguaje de alta complejidad permite una variedad de herramientas a la hora de resolver problemas y comprobación de resultados con fiabilidad, bastante
precisión y en tiempos cortos. \vspace{4mm} 
\\
En los anexos se adjuntarán los codigos en su totalidad para su revisión y los gráficos generados por estos que nos permitiran tener una mayor compresión de los sistemas dinámicos planteados.\vspace{4mm}
\\
Finalmente, en el apartado de conclusiones se hara un balance de los puntos importantes que conllevo la realización del presente informe, habrá especial enfoque en la metodología empleada para la implementación del código.\vspace{4mm}
\\ 
De igual manera en el apartado de refercias se podrán encontrar mayor información con respecto a los temas de sistemas dinámicos y punto de equilibrio de un sistema dinámico.

\section{Propositos}
Los objetivos esperados son los siguientes
\begin{itemize}
    \item Interpretar  el punto de equilibrio del sistema dinámico usando los datos experimentales obtenidos en el laboratorio de robótica
    \item Usar el software de MATLAB para solucionar el sistema dinámico propuesto, haciendo uso de los distintos metodos presentados en sesión de clase.
\end{itemize}
\section{Descripción del Problema}

\section{Solución del Problema}

\section{Resultados}


\section{Conclusiones}

\section{Anexos}

\bibliographystyle{apacite}
\bibliography{biblio}
\end{document}